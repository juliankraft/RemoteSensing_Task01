\documentclass{article}
\usepackage[margin=2.5cm]{geometry} % Set margins
\usepackage{graphicx}
\usepackage[absolute]{textpos} % Enable absolute positioning
\usepackage{titlesec} % Package for controlling section title appearance
\usepackage[scaled]{helvet}
\usepackage[T1]{fontenc}
\usepackage{fancyhdr}

% Set up hyperref
\usepackage{hyperref}
\hypersetup{
    colorlinks=true,
    linkcolor=blue,
    urlcolor=blue,
    linkbordercolor={0 0 1}
}

% Set up references
\usepackage[
    backend=biber,             % Use biber backend (an external tool)
    sorting=none,              % Enumerates the reference in order of their appearance
    style=authoryear           % Choose here your preferred citation style
]{biblatex}
\addbibresource{bibliography.bib} % The filename of the bibliography
\usepackage[autostyle=true]{csquotes} 
                               % Required to generate language-dependent quotes 
                               % in the bibliography

\setlength{\TPHorizModule}{1cm} % Set horizontal unit of measure
\setlength{\TPVertModule}{1cm} % Set vertical unit of measure
\setlength{\parindent}{0pt}

\renewcommand{\familydefault}{\sfdefault}

\makeatletter
\renewcommand{\maketitle}{
  \begin{flushleft} 
    \Large\textmd{\@title} 
    \par
  \end{flushleft}
}
\makeatother

% Define style of sectiontitles
\titleformat{\section}
  {\normalfont\large\mdseries}{\thesection}{1em}{}

% Set up fancyhdr
\fancyhf{} % Clear all headers and footers
\renewcommand{\headrulewidth}{0pt} % Remove the header rule
\rfoot{\thepage} % Place the page number in the right footer
\pagestyle{fancy}


%%%%% Title %%%%%
\title{NDVI calculation for the Kanton of Schaffhausen using Sentinel-2 data and Python}

\begin{document}

%%%%% Header %%%%%
\begin{textblock}{1}(2.5,1) % Position 1cm from left and 1cm from top
        \includegraphics[width=6cm]{logo.jpg} % Add logo
\end{textblock}

\begin{textblock}{6}(13,1) % Position 14cm from left and 1cm from top
        \raggedleft
        Julian Kraft UI22\\
        Remote Sensing\\
        \today
\end{textblock}

\vspace*{1.5cm}

%%%%% Document %%%%%

\maketitle

\section*{Introduction}

The Normalized Difference Vegetation Index (NDVI) is a widely used index to monitor vegetation health and growth. 
It is calculated from the red and near-infrared bands of satellite imagery. The NDVI values range from -1 to 1, 
where values close to 1 indicate healthy vegetation and values close to -1 indicate no vegetation. In this report, 
we calculate the NDVI for the Kanton of Schaffhausen using Sentinel-2 satellite data and Python. To solve this task,
three different Remote Sensing products are used: Sentinel-2 Level-2A data bands B04 (Red) and B08 (NIR) and the
preprocessed Scene Classification Layer (SCL). In this report the focus is on how the data was obtained and processed
and how the created NDVI product looks like for a single time frame.

In next step, namely task 2, the Idea is to analyze the data over time. The NDVI for itself provides limited information
about the actual vegetation health and growth, since the values are relative. Therefore, the plan is to calculate the 
deviation from the mean NDVI over the whole time frame for each pixel on every time frame. 


\section*{Methods}

To process and download the data, the \href{https://openeo.dataspace.copernicus.eu/}{OpenEO} platform was used. There is a
great advantage in using platforms like this, since they provide a lot of functionalities to process date in the cloud
before downloading only the needed result. The platform provides a Python API, which was used to access the data and process it.
The code is available on \href{}{GitHub}. The data processing was done for the are of the Kanton of Schaffhausen
(West: 8.3, South: 47.53, East: 8.9, North: 47.84, ESPG:). 

\section*{Results}


\section*{Discussion}


\vfill
\section*{Declaration of AI Assistance}
ChatGPT was used to solve some problems\\
GitHub copilot was used to assist while writing the report and coding


\end{document}